%%%%%%%%%%%%%%%%%%%%%%%%%%%%%%%%%%%%%%%%%%%%%%%%%%%%%%%%%%%%%%%
%
% Welcome to Overleaf --- just edit your LaTeX on the left,
% and we'll compile it for you on the right. If you open the
% 'Share' menu, you can invite other users to edit at the same
% time. See www.overleaf.com/learn for more info. Enjoy!
%
%%%%%%%%%%%%%%%%%%%%%%%%%%%%%%%%%%%%%%%%%%%%%%%%%%%%%%%%%%%%%%%
\documentclass{beamer}
\usepackage{tikz}
\usetheme{Madrid}
\usepackage{ctex}
\usecolortheme{default}
\usepackage{verbatim}
\title[2022.11.14日汇报] %optional
{2022.11.14汇报}
%\subtitle{Demonstrating the Berkeley theme}
\author[周添文] % (optional)
{周添文\inst{} }

\institute[] % (optional)
{
  \inst{}%
  数学科学学院\\
  北京师范大学\\
 % \and
  %\inst{2}%
%  Faculty of Chemistry\\
%  Very Famous University
}

\date[2022.11.14] % (optional)
%{Very Large Conference, April 2021}

% Use a simple TikZ graphic to show where the logo is positioned

%\node[draw,color=white] at (0,0) {LOGO HERE};


%End of title page configuration block
%------------------------------------------------------------
%The next block of commands puts the table of contents at the 
%beginning of each section and highlights the current section:

\AtBeginSection[]
{
  \begin{frame}
    \frametitle{Table of Contents}
    \tableofcontents[currentsection]
  \end{frame}
}
%------------------------------------------------------------
\begin{document}
\frame{\titlepage}
%\logo{\includegraphics[]{bnu.png}}
%---------------------------------------------------------
%Highlighting text
\begin{frame}
\frametitle{基于Lossy Source Coding和Sparse Transform Matrix的杂散光抑制方法}

\begin{itemize}
\item 杂散光的Point Spread Function(PSF)
\item 有损编码(Lossy Source Coding)
\item 稀疏矩阵变换(STM)
\end{itemize}
%\begin{comment}
%\begin{block}{Remark}
%Sample text
%\end{block}

%\begin{alertblock}{Important theorem}
%Sample text in red box
%\end{alertblock}

%\begin{examples}
%Sample text in green box. The title of the block %is ``Examples".\pause

%HI

%\end{examples}
\end{frame}
\begin{frame}
\frametitle{杂散光的Point Spread Function}
%\begin{itemize}
%\item 
迭代法描述图像还原过程
\begin{equation}
\hat{x}_{t+1}=\hat{x_t}+y-A\hat{x_t}
\end{equation}\pause
其中,$y$是观察到的图像,$\hat{x}_{t}$是第t次迭代后的估计图像,而
\begin{equation}
A=((1-\beta)G_{DA}+\beta S_{STR})
\end{equation}
描述的是杂散光对图像的影响作用\pause

$S_{STR}$是占比为$0<\beta<1$的纯杂散光部分,$G_{DA}$是\textbf{衍射}(Diffraction)和\textbf{像差}(Aberration)部分。

%\end{itemize}
\end{frame}
\begin{frame}
\frametitle{杂散光的Point Spread Function}
在这当中,我们可以写出上述衍射和像差部分在像素$(i_p,j_p)$处的值是
\begin{equation}
G_{DA}(i_q,j_q,i_p,j_p;\sigma)=\frac{K}{\sigma ^2}\exp(-\frac{(i_q-i_p)^2+(j_q-j_p)^2}{2\sigma ^2})
\end{equation}
其中,K是正规化参数\pause

由此可得,纯杂散光部分的表达式是
\begin{align*}
S_{STR}(i_q,j_q,i_p,j_p;a,b,c,\alpha)&=
\frac{1}{z}(1+\frac{1}{i_p^2+j_p^2}(\frac{(i_qi_p+j_qj_p-i_p^2-j_p^2)^2}{(c+a(i_p^2+j_p^2))^2}\\&+\frac{(-i_qi_p+j_qj_p)^2}{(c+b(i_p^2+j_p^2))^2})
\end{align*}

\end{frame}
\begin{frame}
\frametitle{杂散光的Point Spread Function}
其中$a,b,c,z,\alpha$均为正规化参数,其存在如下关系:
\begin{equation}
z=\frac{\pi}{\alpha-1}(c+a(i_p^2+j_p^2)(c+b(i_q^2+j_q^2))
\end{equation}
使得上式满足
\begin{equation}
\int_{i_q}\int_{j_q}S_{i_q,j_p}di_qdj_q=1
\end{equation}\pause
由文献[2]中的\textbf{余弦四次方定律}(cosine fourth law),我们知道模型整体需要乘上$cos^2\gamma=\frac{D^2}{i_p^2+j_p^2+D^2}$,D为出瞳位置到图像平面的距离,$\gamma$表示图像系统的光轴和光所成的角度\\
\end{frame}
\begin{frame}
\frametitle{杂散光的Point Spread Function}
综上,杂散光的PSF模型总体可描述为:
\begin{equation}
S=\frac{D^2}{i_p^2+j_p^2+D^2}((1-\beta)G_{DA}+\beta S_{STR})
\end{equation}\pause

因此,我们得到了图像还原的简化过程:
\begin{equation}
\hat{x}=2y-(1-\beta)y-\beta Sy
\end{equation}\pause

在这当中,$Sy$是与空间位置(像素位置)有关系的量,因此在面对真实图片时,其计算量很大, 故而我们需要引入\textbf{有损编码}(Lossy  Source Coding)进行计算的简化。
\end{frame}
\begin{frame}
\frametitle{有损编码(Lossy Source Coding)}
这一技术的目的是将图片的信息集中,以此简化计算

这一部分运算比较复杂,\alert{细节没有完全理解},其整体分为如下几步:\pause

\begin{itemize}
\item 测量白化(Measurement Whitening)
\item 将矩阵$S$的列向量去相关化
\item 小波变换(wavelet transform)
\item 离散化
\end{itemize}
\end{frame}
\begin{frame}
\frametitle{稀疏矩阵变换(Sparse Matrix Transform)}
稀疏矩阵变换是将一个矩阵T分解成一系列稀疏矩阵(含有大量0元素)的乘积的过程,即
\begin{equation}
T=\prod_{k=K-1}^{0}T_k=T_{k-1}T_{k-2}\cdots T_0
\end{equation}\pause
而每一个稀疏矩阵$T_k$都可以分解为$T_k=B_k\Lambda _k A_k$,其中$A_k,B_k$为旋转矩阵,$\Lambda _k$是单位矩阵
\end{frame}
\begin{frame}
\frametitle{算法展示}
\subsection{算法}
\textbf{输入}:观察到的图像$y$

\textbf{输出}:修正后的图像$\hat{x}$

\textbf{初始化}:设置参数$a,b,c,\alpha,\beta,D$的值\pause

\begin{itemize}
\item 对原始图像应用PSF模型
\begin{equation}
S=\frac{D^2}{i_p^2+j_p^2+D^2}((1-\beta)G_{DA}+\beta S_{STR})
\end{equation}\pause
\item 应用Lossy source coding,将矩阵S转化为易分解为稀疏矩阵的形式\pause
\item 应用稀疏矩阵变换\pause
\item 得到$X=Sy$的表达式,带入原方程
\begin{equation}
\hat{x}=2y-(1-\beta)y-\beta X
\end{equation}
\end{itemize}
\end{frame}
\begin{frame}
\frametitle{基于稀疏约束和多尺度的杂散光抑制方法}
该方法主要采用求解偏微分方程的方法,细节较为复杂,还没有阅读完成,其总体思路如下:

将观察到的退化图像$Z$分解为目标图像$I$和杂散光图像$B$,其关系为
\begin{equation}
Z_i=I_i+B_i+n, 1\leq i\leq n_p
\end{equation}\pause
其中,$n_p$是图像中像素个数,$i$为像素的序号,$n$是噪声的影响。而整个杂散光抑制的过程则是\textbf{在给定的退化图像$Z$的基础上,估计式子中杂散光图像$B$,进而估计$I$。}
\end{frame}
\begin{frame}
\frametitle{基于稀疏约束和多尺度的杂散光抑制方法}
我们对杂散光图像$B$和目标图像$I$添加正则约束$C_I(I), C_B(B)$,则杂散光抑制问题转化了如下的能量函数最小值问题:
\begin{equation}
F(\hat{I},\hat{B})=\arg\min \frac{1}{2}||Z-I-B||_2^2+\alpha C_B(B)+\beta C_I(I)
\end{equation}
其中$\alpha >0, \beta>0$是平衡约束项的参数\pause

最后,我们只需要用\textbf{Euler-Lagrange法}求解上述泛函的极值即可。
\end{frame}
\begin{frame}
\frametitle{参考文献}
\begin{thebibliography}{99}
\bibitem{article1}P.A. Jansson, Deconvolution of Images and Spectra, Springer, New York, 1996.
\bibitem{article2}W.J. Smith, Modern Optical Engineering: The Design of Optical Systems,
McGraw Hill, New York, 2000.
\bibitem{article3}李思圆. 基于稀疏约束的杂散光抑制算法[D].华中科技大学,2016.
\bibitem{article4}Fan Y H, Qin S Y. A fast algorithm for stray light correction based on lossy source coding and SMT[J]. Optik, 2013, 124(14): 1677-1682.
\end{thebibliography}
\end{frame}
\end{document}